\documentclass{hbrs-ecta-report}

\usepackage{float}
\usepackage{placeins}
\usepackage{algorithm}
\usepackage{algorithmicx}
\usepackage{algpseudocode}

\begin{document}

\conferenceinfo{H-BRS}{2017}

\title{NeuroEvolution of Augmenting Topologies (NEAT)}
\subtitle{Steuerung eins Mario Jump \& Run Spiels}

\numberofauthors{2}
\author{
Jan Urfei\\
       \affaddr{Bonn-Rhein-Sieg University of Applied Sciences}\\
       \affaddr{Grantham-Allee 20}\\
       \affaddr{53757 Sankt Augustin, Germany}\\
       \email{jan.urfei@inf.h-brs.de}\\
\and
	Tim Lügger\\
	\affaddr{Bonn-Rhein-Sieg University of Applied Sciences}\\
	\affaddr{Grantham-Allee 20}\\
	\affaddr{53757 Sankt Augustin, Germany}\\
	\email{tim.luegger@inf.h-brs.de}\\
}
\date{today}
\maketitle
\begin{abstract}
Ziel war es eine Mario Spielfigur mit bestimmten Szenarien zu trainieren, sodass diese dann das/die Level best möglichst lösen kann.
\end{abstract}

%% ------------------------------ GENERAL NOTES ------------------------------ %%
\section{General Notes}
\label{sec:generalnotes}
\begin{itemize}
\item Do not state things you cannot confirm either by literature or experiment. This is science, not black magic.
\item Normalize your data and resulting statistical descriptors (like RMSE, $\mu$)
\item Save your experimental results to disk \textbf{before you visualize}. You \textbf{will} change your plots quite a lot after gathering the data.
\item Don't submit reports with many pages containing just a single figure. Scrolling is terrible for our health.
\item \textbf{Separate} training and test data. Don't use test data in your algorithm to learn or make decisions, they are only allowed to test the end result. If you need a separate sample set to make decisions during training or optimization, create a third (validation) set. Only your test data will really tell you how good your algorithm is. 
\end{itemize}

\subsection{Algorithm Parametrization}
Develop an actual strategy, preferably on a reduced but similar problem. You can reduce the number of samples, the targetted number of time steps your controller runs in a simulation, or any other non-destructive problem reduction method. If your algorithm has two paramers you need to adjust, it should be no problem to take 5 \textit{sensible} values per parameter and compare all combinations. For stochastic algorithms, like evolutionary approaches, make sure you \textbf{repeat your experiments} at least 5-10 times, depending on the amount of randomness. Since your algorithm makes "random" changes, just comparing single runs does not give you a good estimation on its performance. Do \textbf{not} pick your values such that they only confirm the values you \textbf{want} to use.

%% ------------------------------ STRUCTURE ------------------------------ %%
\subsection{Structure}
\label{sec:structure}
\begin{enumerate}
\item \textbf{Assignment Description}: first provide a brief description of the assignment that was handed out to you. This includes a description of the data.
\item \textbf{Approach}: describe the algorithm
\item \textbf{Experiments}: describe your experimental setup, algorithm parameters, data preprocessing first. Then include the results and a discussion thereof.
\item \textbf{Conclusion}: any conclusions about the algorithm, bugs, future work.
\end{enumerate}
%% ------------------------------ VISUALIZATION ------------------------------ %%
\subsection{Visualization}
\label{sec:visualization}

\begin{itemize}
\item Label your axes
\item Use logarithmic scale when appropriate
\item Use descriptive captions below your figures
\item Add legends if necessary
\item Make fontsizes large enough, linewidths thick enough to be readible in the final report.
\item When making comparisons, make sure the results are either in the \textbf{same} graph or graphs are plotted next to (or close to) each other.
\item \textbf{In short: make sure people can read your figures}
\end{itemize}

%% ------------------------------ REPORT STRUCTURE ------------------------------ %%
\FloatBarrier
\newpage
\newpage
\section{Assignment}
Aufgabe war es den für die HeartRate Prediction implementierten NEAT Algorithmus\cite{Stanley2002a} anzupassen, dass er ein Problem eines ausgesuchten Projektes löst.

In diesem Projekt wurde ein Simulator verwendet, der dem Jump\&Run Spiel Super Mario Bros. nachempfunden ist.
%todo cite simulator lib
Ziel diese Spieles ist eine Figur in einer 2D-Welt zu steuern und möglichst das Ende des Levels zu erreichen. Dabei existieren Hindernisse wie Schluchten oder Monster die es zu überwinden gilt. 

Zielsetzung war es zu ergründen, ob es NEAT erlaubt ein Netzwerk zu trainieren, welches basierend auf den Informationen aus dem aktuellen Sichtfeld die Spielfigur zu steuern, um ans Ende des (statischen) Levels zu gelangen.

Als Trainingsdaten sich zunächst auf ein generiertes Level beschränkt, welches nur Blöcke, Schluchten oder erhöhte Ebenen enthielt, die es zu überspringen galt.

Das Sichtfeld beseht aus einem Gitter, zentriert um die Spielfigur herum (siehe Figure \ref{fig:MarioInput}). Die Objekte/Kacheln im Level unterscheiden sich anhand ihrem Typ (z.B Boden, Wand) und ihrer Position.

\begin{figure}[h!]
	\centering
	\includegraphics[width=\linewidth]{img/MarioInput.jpg}
	\caption{Umgebungsblöcke der Spielfigur für den Input}
	\label{fig:MarioInput} 
\end{figure}

\FloatBarrier
\section{Approach}
Im Gegensatz zum NEAT Paper \cite{Stanley2002a} wird eine Mutation verwendet, die Kanten wieder deaktivieren kann, damit die Topologien minimiert werden, falls nicht alle Eingaben aus dem Sichtfeld benötigt werden. 
Des weiteren wurde auf Fitnesssharing verzichtet.


Um die Kommunikation zwischen dem Simulator und Matlab herzustellen wurde die MATLAB API for Java verwendet. Diese erlaubt es aus Java heraus mit der Matlab Seesion zu interagieren.
In jeder Iteration wird die aktuelle Population aus dem Workspace geladen und bei der Ausführung des Simulators anhand des aktuellen Sichtfelds der Spielfigur der Output bzw. den daraus resultierenden Tastendruck. 
Um die Kommunikation zwischen Matlab und Java möglichst gering zu halten ist die Berechnung der Netze in Java mittels der Matrizen Bibliothek NDJ4 %todo cite
implementiert. 
Hat der die Spielfigur das Ende des Levels erreicht, ist die Spielzeit abgelaufen oder die Spielfigur in eine Schlucht gefallen wird die fitness auf den erreichten Levelfortschirtt gesetzt (x-Position). Außerdem wird bei Topologien, die das Ende des Levels erreichen, die restliche Spielzeit addiert um die Geschwindigkeit in der das Level abgeschlossen wird zu belohnen. Die Fitnesswerte werden zurück in den Matlab Workspace geschrieben und die Mutation mittels NEAT durchzuführen. Danach beginnt der Prozess von neuem. 
In der folgenden Abbildung (Figure \ref{fig:sequenzdiagramm}) ist der Ablauf nochmals veranschaulicht.

\begin{figure}[h!]
	\centering
	\includegraphics[width=\linewidth]{img/JavaMatlab.png}
	\caption{Kommunikation und Ablauf zwischen Java und Matlab}
	\label{fig:sequenzdiagramm} 
\end{figure}

Das Sichtfeld der Spielfigur ist auf 15x15 (Breite x Höhe) Felder begrenzt bzw. auf 7x15 also nur die rechte Hälfte des Sichtfeldes, da im ausgewählten Level keine Rückwertsbewegung notwendig ist.

Um die Eingabemenge gering zu halten wurden ähnliche Kacheln mit dem selben Eingabewert versehen (z.b Boden, Fels) sowie unwichtigeren Kacheln geringere Eingabegrößen vergeben.

\FloatBarrier
\section{Experiments}
Experiments are repeated 10 times. By the way, this is not enough for a description.

\subsection{Parameterization}
This would include a description of all parameter tuples and a description on how you reduced the problem to provide small and fast runs for this extensive comparison. Results would be shown in boxplots, as these provide you with the median and 25\% and 75\% percentile for every parameter tuple. This makes comparison easy, as you can plot multiple boxplots next to each other, similar as in Figure \ref{fig:4}.

\subsection{NEAT vs ESP}

\begin{figure}[ht!]
\centering
\includegraphics[width=\linewidth]{img/1.png}
\caption{Development of average mean square error, comparing NEAT and ESP}
\label{fig:1} 
\end{figure}
 
\begin{figure}[ht!]
\centering
\includegraphics[width=\linewidth]{img/2.png}
\caption{Development of min/mean/max number of nodes and edges. \textit{Legend is missing}!}
\label{fig:2} 
\end{figure}

\begin{figure}[ht!]
\centering
\includegraphics[width=\linewidth]{img/3.png}
\caption{Elite topology. Nodes are assigned their IDs. Green are input nodes, purple are output nodes.}
\label{fig:3} 
\end{figure}

\begin{figure}[ht!]
\centering
\includegraphics[width=\linewidth]{img/4.png}
\caption{Comparing MSE of NEAT and ESP}
\label{fig:4} 
\end{figure}

\begin{figure}[ht!]
\centering
\includegraphics[width=\linewidth]{img/5.png}
\caption{Approximated heart rates after 300 generations}
\label{fig:5} 
\end{figure}

\begin{figure}[ht!]
\centering
\includegraphics[width=\linewidth]{img/6.png}
\caption{Development of species over time, averaged over 10 generations}
\label{fig:6} 
\end{figure}


\FloatBarrier
\section{Conclusion}


\bibliographystyle{abbrv}
\bibliography{HeteroNEAT} 
\end{document}
}